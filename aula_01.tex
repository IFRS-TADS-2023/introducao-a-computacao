\documentclass{article}
\usepackage{indentfirst}

\author{Arthur Von Groll dos Santos}
\title{Aula 01}

\begin{document}
\maketitle

\section*{\centering 1º Período}

\begin{itemize}
    \item Introdução da disciplina
\end{itemize}

\section*{\centering 2º, 3º e 4º Períodos}

\subsection*{\centering Dados e Informações}

Dados são como números aleatórios, não tem significado; já informações tem significado.

Exemplo: Se alguém entrar em uma sala de aula e falar `vinte e cinco', essa pessoa está apresentando
um dado. Por outro lado, se ela falar `no dia 25 não teremos aula', está passando uma informação.

\subsection*{\centering Hardware x Software}

Hardware é físico (ex: HD, CPU, impressora, teclado), já o software é lógico (ex: Windows, Word, Excel).

\subsection*{\centering Hardware}
\begin{itemize}
    \item \textbf{Entrada}: Onde se recebe dados. Envia os dados para a unidade de 
        processamento.
    \item \textbf{Processamento}: Onde os dados são transformados em informação, normalmente
        passando por circuitos eletrônicos
    \item \textbf{Armazenamento}: Dispositivos de armazenamento secundários, como 
        disco rígido (HDD, SSD)
    \item \textbf{Saída}: Onde os dados processados (as informações) são exibidos 
        de uma forma compreensível e conveniente
\end{itemize}

\subsection*{\centering Componentes de Hardware}

\begin{itemize}
    \item \textbf{CPU (Central Processing Unit)}: Comanda periféricos 
        por meio de comandos diretos ou para sua interface.
    \begin{itemize}
        \item \textbf{Unidade de Controle}: É a parte do processador que controla o 
            ciclo da máquina. Serve para controlar as atividades das demais unidades
            do sistema.
        \item \textbf{Unidade Lógica Aritmética}: É a parte do processador que executa
            as operações matemáticas.
    \end{itemize}

    \item \textbf{Memória}: Dispositivo de armazenamento que visa maior rendimento.
        É associado com a CPU. A memória é altamente veloz, porém é \emph{volátil}
        (perde os dados ao reiniciar a máquina). Uma unidade de memória é uma coleção de células binárias,
        capaz de armazenar uma grande quantidade de informação binária.
\end{itemize}

\end{document}